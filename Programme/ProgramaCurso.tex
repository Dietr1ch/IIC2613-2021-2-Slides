\documentclass[11pt,letterpaper]{article}
\usepackage{enumerate}
\usepackage[utf8x]{inputenc}
\usepackage{graphicx}	
\usepackage[letterpaper,margin=0.7in,includefoot]{geometry}

\graphicspath{{./Figures}}


\begin{document}

\hbox
{
  \centerline
  {
    \vbox
    {
      \hbox{\sc\Large Pontificia Universidad Cat\'olica de Chile}\vspace{1ex}
      \hbox{\sc\Large Escuela de Ingenier\'ia}\vspace{1ex}
      \hbox{\sc\Large Departamento de Ciencia de la Computaci\'on}
    }
    \raisebox{-4pt}{\includegraphics[width=0.14\textwidth]{logo}}
  }
}
\centerline
{
    \rule{480pt}{1pt}    
}

\begin{center}
\vspace{2ex}\Large\bf IIC2613 - INTELIGENCIA ARTIFICIAL \\
\vspace{1ex}-- Programa de curso --
\end{center}

\vspace{4ex}
% Info administrativa
\begin{center}
\begin{tabular}{l@{\quad:\quad}l}
  {\bf Profesores} & 
Dietrich Daroch (\verb+ddaroch@ing.puc.cl+) y Álvaro Soto
(\verb+asoto@ing.puc.cl+)\\  
  {\bf Requisitos} & IIC2233\\  
  {\bf Sitio Web} & SIDING\\  
  {\bf Clases} & Lunes y Miércoles, módulo 2 \\
  {\bf Ayudantías} & Viernes, módulo 2 \\
  {\bf Horario de atención} & enviar email al profesor para concertar una cita\\  
\end{tabular}
\end{center}
\vspace{2ex}

% Contenido
\section{Presentación del curso}
El objetivo principal del curso es que el alumno comprenda los conceptos
fundamentales relacionados con el área de Inteligencia Artificial y las
metodologías que se utilizan en ésta. En particular, el alumno aprenderá a
aplicar técnicas clásicas para la resolución de problemas usando lógica
deductiva, algoritmos de búsqueda y técnicas de planificación. Además aprenderá
los principios básicos del área de aprendizaje de máquina junto con sus
principios inductivos de inferencia, revisando algunas de las técnicas más
utilizadas de esta creciente área.

\section{Objetivos de aprendizaje}
Al finalizar el curso los alumnos serán capaces de:
\begin{itemize}
  \item Entender la evolución histórica de la inteligencia artificial, en
    particular sus 2 ramas principales de inferencia inductiva e inferencia
    deductiva.
  \item Entender y aplicar técnicas deductivas de inteligencia artificial.
  \item Enterder y aplicar técnicas inductivas de inteligencia artificial.
  \item Analizar complejidad computacional y requerimientos de memoria asociados
    a la aplicación de técnicas de inteligencia artificial.
  \item Analizar problemas que requieran el uso de técnicas de inteligencia
    artificial y crear soluciones acordes basadas en el paradigma de un agente
    inteligente.
\end{itemize}

\section{Contenido}
A continuación se presenta un desglose detallado de los contenidos del curso:

\begin{enumerate}
  \item \textbf{Introducción}    
  \item \textbf{Resolución de problemas mediante búsqueda}
    \begin{itemize}
    \item Formalización de problemas de búsqueda.
    \item Búsqueda no informada (DFS, BFS, Dijkstra's Algorithm).
    \item Búsqueda informada (A*, IDA*).
    \item Búsqueda en juegos (Minimax, Monte Carlo tree search).
    \end{itemize}
  \item \textbf{Programación en Lógica:}
    \begin{itemize}
      \item Answer Set Programming.
      \item Aplicaciones en Planificación y Diagnóstico.            
    \end{itemize}
  \item \textbf{Introducción al aprendizaje de máquina}
    \begin{itemize}    
      \item Conceptos básicos.
      \item Tipos de aprendizaje.      
      %\item Optimización continua.
    \end{itemize}
  \item \textbf{Aprendizaje supervisado y reforzado}
    \begin{itemize}                  
      \item Máquinas de vectores de soporte (SVM).
      \item Árboles de decisión y Random Forest.
      \item Naive Bayes.
      \item Redes neuronales.
      \item MDP y aprendizaje reforzado.
      \item Intro a redes neuronales profundas (Deep Learning).
    \end{itemize}              
\end{enumerate}

\section{Metodología}
El curso se desarrollará en clases expositivas de 80 minutos de duración. El
alumno deberá rendir controles escritos y elaborar tareas en forma individual
para complementar su aprendizaje. Los apuntes del curso, enunciados de tareas y
pautas de corrección de controles estarán disponibles en forma electrónica en el
sitio web del curso.

\section{Evaluaciones}
Las evaluaciones se dividen en dos tipos, cada una con su correspondiente nota
final promedio:

\begin{itemize}
 \item Actividades en clases (30\%): corresponde a controles escritos cortos
   sobre los contenidos de clases previas. Se realizará a lo más un control por
   clase. Se espera aplicar al menos 10 controles durante el semestre, los
   cuales serán en fechas aleatorias. Los alumnos podrán dejar de rendir
   controles sin necesidad de justificación, por lo cual, en el cálculo de la
   nota final de controles ($C$) se eliminarán las 3 notas más bajas.

\item Tareas de programación (70 \%): Se realizarán 4 tareas de igual valor y de
  carácter sumativo, sobre tópicos visto en cátedra, desde la óptica de la
  programación.

La nota final se calcula de la siguiente manera: $\textbf{NF} = 0.3 * C + 0.7 *
\frac{T_1 + T_2 + T_3 + T_4}{4}$
\end{itemize}



\section{Normativas adicionales}
\begin{itemize}
 \item El descuento por entrega de tareas atrasadas es de 0.5 punto por día de
   atraso. Con un máximo de 1 semana de atraso.
 \item Una vez publicada la nota respectiva, cada evaluación tendrá un período
   m\'aximo de 1 semana para la entrega de solicitudes de recorrección. \'Estas
   deben ser enviadas al ayudante jefe.
\end{itemize}


\section{Política de Integridad Académica}
Los alumnos de la Escuela de Ingeniería deben mantener un comportamiento acorde
al Código de Honor de la Universidad:

\begin{flushleft}
\textit{``Como miembro de la comunidad de la Pontificia Universidad Católica de
  Chile me comprometo a respetar los principios y normativas que la rigen.
  Asimismo, prometo actuar con rectitud y honestidad en las relaciones con los
  demás integrantes de la comunidad y en la realización de todo trabajo,
  particularmente en aquellas actividades vinculadas a la docencia, el
  aprendizaje y la creación, difusión y transferencia del conocimiento. Además,
  velaré por la integridad de las personas y cuidaré los bienes de la
  Universidad.''}
\end{flushleft}

En particular, se espera que mantengan altos estándares de honestidad académica.
Cualquier acto deshonesto o fraude académico está prohibido; los alumnos que
incurran en este tipo de acciones se exponen a un procedimiento sumario.
Ejemplos de actos deshonestos son la copia, el uso de material o equipos no
permitidos en las evaluaciones, el plagio, o la falsificación de identidad,
entre otros. Específicamente, para los cursos del Departamento de Ciencia de la
Computación, rige obligatoriamente una política de integridad académica en
relación a copia y plagio. Si un alumno copia un trabajo, se le calificará con
nota 1.0 en dicha evaluación y dependiendo de la gravedad de sus acciones podrá
tener un 1.0 en todo ese ítem de evaluaciones o un 1.1 en el curso. Además, los
antecedentes serán enviados a la Dirección de Pregradp de la Escuela de
Ingeniería para evaluar posteriores sanciones en conjunto con la Universidad,
las que pueden incluir un procedimiento sumario. Por ``copia'' o ``plagio'' se
entiende incluir en el trabajo presentado como propio, partes desarrolladas por
otra persona. Está permitido usar material disponible públicamente, por ejemplo,
libros o contenidos tomados de Internet, siempre y cuando se incluya la cita
correspondiente.

\section{Bibliografía}
\begin{itemize}
  \item S. Russell, P. Norvig, \textit{Artificial Intelligence, A Modern Approach}, Prentice Hall, 3rd edition, 2010.
  \item I. Goodfellow, Y. Bengio, A. Courville. \textit{Deep Learning}, MIT Press, 2016.
  \item L. Sterling, E. Shapiro. \textit{The Art of Prolog}, MIT Press, 1994.
  \item C. Bishop, \textit{Pattern Recognition and Machine Learning}, Springer, 2006.
  \item T. Mitchell, \textit{Machine Learning}, McGraw Hill, 1997.
  \item R. Duda, P. Hart, D. Stork, \textit{Pattern Classification}, Wiley Interscience, 2nd edition, 2000.
  \item T. Hastie, R. Tibshirani, J. Friedman, \textit{The elements of Statistical Learning}, Springer, 2nd edition, 2009.
\end{itemize}

\end{document}